\chapter{How Fast is the Thymio?}\label{ch.fast}

There are three related projects in this chapter. The first project measures the
speed of the robot by measuring how long it takes the robot to move a
fixed length. The second project enhances the first one by enabling you
to change the speed without making a change in the program. The third
project measures the acceleration of the robot.

\section{Measuring the speed of the Thymio}\label{s.measure}

\sect{Specification}

Measure the speed of the Thymio robot.

\sect{System design}

Construct a long, narrow rectangle of black tape on the floor and place
the robot near the left end of the rectangle facing right:\footnote{I am
assuming that the floor has a light color. If the floor has a dark
color, you should use white tape and change the program accordingly.}

\begin{center}
\begin{picture}(315,45)
%\put(0,0){\framebox(315,45){}}
\put(0,2){
\put(0,0){\framebox(30,30){}}
\put(30,15){\oval(30,30)[r]}
\put(45,15){\vector(1,0){15}}
}
\put(65,5){\colorbox{black}{\makebox(200,24){}}}
\put(75,12){\colorbox{white}{\makebox(180,10){}}}

\put(135,37){\vector(-1,0){60}}
\put(195,37){\vector(1,0){60}}

\put(146,33){\makebox(40,10){\textsf{1 meter}}}

\put(277,17){\vector(0,1){8}}
\put(277,17){\vector(0,-1){8}}
\put(282,14){\makebox(40,10)[l]{\textsf{$\sim$ 10 cm}}}

\end{picture}
\end{center}

The distance between the right edge of the left tape and the left edge
of the right tape should be exactly one meter, while the distance
between the top and bottom tapes should be sufficient so that both
ground sensors can detect the floor, but a relatively small turn will
cause the robot to detect the tapes.

When the forward button is touched, the robot moves slowly straight
ahead until it detects the left tape; it continues to move over the tape
until it detects the right edge of the left tape. At that point, it
starts moving rapidly until it reaches the left edge of the right tape,
at which point it stops. If the robot detects the top or bottom tapes,
it turns back to a straight course.

When the robot detects the right edge of the left tape, it sets a time
counter to zero. The value of the time counter when the robot stops is
used to compute the time that the robot takes to travel one meter and
thus its velocity.

Touching the center button when the robot is moving causes it to stop.

\sect{State machine}

The program transitions sequentially through four states:

\begin{center}
\unitlength=1.2pt
\begin{picture}(300,35)
%\put(0,0){\framebox(300,35){}}
\put(30,10){\oval(60,20)}
\put(110,10){\oval(60,20)}
\put(190,10){\oval(60,20)}
\put(270,10){\oval(60,20)}
\put(0,0){ \makebox(60,20){\bu{0=off}}}
\put(80,0){\makebox(60,20){\bu{1=find}}}
\put(160,0){\makebox(60,20){\bu{2=skip}}}
\put(240,0){\makebox(60,20){\bu{3=straight}}}
\put( 60,10){\vector(1,0){20}}
\put(140,10){\vector(1,0){20}}
\put(220,10){\vector(1,0){20}}
\put(270,20){\line(0,1){15}}
\put(30,35){\line(1,0){240}}
\put(30,35){\vector(0,-1){15}}
\end{picture}
\end{center}

\begin{itemize}

\item The transition from \bu{off} to \bu{find} occurs when
the center button is touched.

\item The transition from \bu{find} to \bu{skip} occurs
when the left edge of the left tape is detected.

\item The transition from \bu{skip} to \bu{straight} occurs
when the right edge of the left tape is detected.

\item The transition from \bu{straight} to \bu{off} occurs
when the left edge of the right tape is detected.

\item Additionally, in all the states except \bu{off}, touching the
center button causes a transition to state \bu{off}. (These transitions
are not shown in the diagram above.)
\end{itemize}

The value of the current state is displayed by turning on a LED
associated with a button: 0=front, 1=right, 2=rear, 3=left.


\sect{Constants}

\begin{itemize}
\item \p{THRESHOLD}: The threshold below which a sensor detects that it
is on a tape.

\item \p{MOTOR}: The power setting of the motors.

\item \p{CHANGE}: The percentage change in the motor power when turning.
\end{itemize}

\sect{Variables}

\begin{itemize}
\item \p{state}: The current value of the state machine (0 .. 3). 
\item \p{time}:  A counter of tenths of seconds.

\end{itemize}

\sect{Event handlers and subroutines}

\begin{itemize}

\item Initialization: Initialize \p{state} to 0, set the LEDs
accordingly, initialize \p{timer0} to occur every 100 milliseconds = 0.1
seconds, and turn the motors off.

\item When the center-button event occurs: if the button is released,
the subroutine \p{stop} is called. It sets the state to 0 (\bu{off}) and
initializes the motors.

\item When the \bu{timer0} event occurs: if the current state is 3
(\bu{straight}), the variable \p{time} is incremented.

\item When a proximity event occurs, the subroutine called depends on
the state:

\begin{itemize}

\item In state 1 (\bu{find}) subroutine \p{start\_found} is called. If
the start of the tape is found (both ground sensors sense black---a
value less than the threshold), the motors are set to run at a reduced
speed. The state is changed to 2 (\bu{skip}).

\item In state 2 (\bu{skip}) subroutine \p{end\_of\_start\_found} is
called. When both ground sensors no longer sense black, this is the end
of the start tape. The state is changed to 3 (\bu{straight}), the motors
are set to run at the value \p{MOTOR} and the variable \p{time} is
initialized to 0.

\item In state 3 (\bu{straight}) subroutine \p{drive\_straight} is
called. If both ground sensors sense black, this is the finish line and
\p{stop} is called. If only one sensor senses black, the robot is turned
to get it going straight again by increasing the speed of one wheel and
decreasing the other. If neither sensor senses black, the robot continues
straight.

\end{itemize}
\end{itemize}

\sect{Programming notes}

{\raggedleft \hfill Program file \bu{speed1.aesl}}

The initialization statements and those in the subroutine \p{stop} are
almost the same, but have to be repeated because the initialization
cannot call a subroutine.

The constant \p{CHANGE} specifies the percentage by which the motor
power setting \p{MOTOR} should be increased or reduced to turn the
robot. The percentage 25\% is expressed in mathematical calculation by
the decimal number $0.25$, but Aseba does not support decimal numbers.
If the value of \p{CHANGE} is 25, we cannot write \p{CHANGE / 100},
because division in Aseba is \emph{integer division}: $25/100$ equals 0
with remainder 25.\footnote{You can obtain the remainder using the
\emph{modulo} operation in Aseba: \p{CHANGE \% 100}.} The solution is to
first perform a multiplication and then a division, so, for example,
$(300\times 25)/100 = 75$. The Aseba statements to change the power
setting are:

\begin{verbatim}
  motor.left.target  = MOTOR + (MOTOR * CHANGE)/100
  motor.right.target = MOTOR - (MOTOR * CHANGE)/100
\end{verbatim}

\centeredbox{The value of the variable \p{time} when the robot stops can
be obtained by looking in the table labeled \textbf{Variables} at the
left of the Aseba Studio window. When the robot stops, click
\bu{refresh}. Better, click the check box \bu{auto} and the display will
be dynamically refreshed.\label{p.variables}}

Use the top circle LEDs to display the value of the variable \p{time}.
Since there are only 8 LEDs the display can only updated at long
intervals, for example, every 4 seconds to cover the range from 0
through 32 seconds. Remember that the variable \p{time} counts in tenths
of a second, so the subroutine \p{set\_circle\_leds} is:\footnote{31
is used instead of 32 to prevent the overflow of the intensity value.}
\begin{verbatim}
  sub set_circle_leds
    call leds.circle(
      (time/ 40)*31, (time/ 80)*31, (time/120)*31, (time/160)*31,
      (time/200)*31, (time/240)*31, (time/280)*31, (time/310)*31)
\end{verbatim}  

\sect{Measuring the speed}

The robot was run five times, one for each setting of the constant \p{MOTOR}: 100, 200, 300,
400, 500. The value of the variable \p{time} was recorded for each run
and divided by ten since the variable records tenths of a second.
Dividing 100 centimeters (= 1 meter) by the time gives the (average)
speed of the robot:

\begin{center}\label{tab.speed}
\begin{tabular}{|c|r|r|}\hline
Power  & Time (sec)  &Speed (cm/sec)\\\hline
100&30.5  &3.3\\\hline
200&15.5  &6.5\\\hline
300&10.5  &9.5\\\hline
400&7.8   &12.8\\\hline
500&6.5   &15.6\\\hline
\end{tabular}
\end{center}

Here is a graph of power (x-axis) vs. speed (y-axis):

\gr{accel}{.6}

The relation is linear! Linearity is a desirable property because you
can easily change the power to obtain any desired speed. If the relation
were not linear, you would have to adjust the power setting by including
a table of power vs. speed in the program, as described in
Chapter~\ref{ch.nonlinear}.

\section{Setting the power from the Thymio}

\sect{Specification}

It is difficult to measure the speed of the robot for different power
settings because for each one you have to change the program and
reconnect the robot to the computer to load the new program. Modify the
program so that the power can be changed by pressing buttons on the
robot.

\sect{System design}

The right and left buttons are used to increase and decrease the power
setting in the range 0 .. 500. The setting is displayed by turning on 0
.. 5 circle LEDs, given by the power setting divided by 100.

\sect{Constants}

\begin{itemize}
\item The constant \p{MOTOR} is no longer needed.
\item \p{INCREMENT} is the amount by which the motor power is increased
or decreased when the right and left buttons are pressed.
\end{itemize}

\sect{Variables}

\begin{itemize}
\item \p{motor}: The motor power setting.
\item \p{motor\_change}: The change to the motor power setting.
\end{itemize}


\sect{Event handlers and subroutines}

Two event handlers \p{button.left} and \p{button.right} are needed to
detect when these buttons are released.

The subroutine \p{set\_circle\_leds} lights the correct number of
circle LEDs.

\sect{Programming notes}

{\raggedleft \hfill Program file \bu{speed2.aesl}}

All occurrences of the constant \p{MOTOR} must be replaced by the
variable \p{motor}.

In order to avoid recomputing the expression \p{(motor * CHANGE)/100}
frequently, this value is assigned to the variable \p{motor\_change}
whenever the value of \p{motor} is changed.

\begin{verbatim}
  motor_change = (motor * CHANGE)/100
\end{verbatim}
The assignment of power setting to the power uses this value:
\begin{verbatim}
  motor.right.target = motor - motor_change
  motor.right.target = motor - motor_change
\end{verbatim}

When increasing or decreasing the motor power, we have to be careful not
to give values that are outside the range 0 .. 500 that the motor
accepts. To avoid \emph{saturation}, the range is checked when the
left and right buttons are touched:

\begin{verbatim}
  if  motor < 0 then motor = 0 end
  if  motor > 500 then motor = 500 end
\end{verbatim}

The subroutine \p{set\_circle\_leds} turns on 0 through 5 LEDS
depending on the value of \p{motor}:\label{p.leds}

\begin{verbatim}
  sub set_circle_leds
    call leds.circle(
      (motor/100)*32, (motor/200)*32, (motor/300)*32,
      (motor/400)*32,(motor/500)*32, 0,0,0)
\end{verbatim}
for each value in the range 0 .. 5. Division in Aseba is integer
division with truncation so if the value of \p{motor} is 328, the result
of the division is 3.

\section{Measuring acceleration of the Thymio}

Acceleration is the rate of change of the speed. For example, if the
speed of the robot starts at 0, becomes 5 cm/sec after 1 second, then 10
cm/sec after 2 seconds, and finally 15 cm/sec after 3 seconds, the
acceleration is 5 cm/sec per second, written 5 cm/sec$^2$.

\pagebreak[4]

\sect{Specification}

Measure the acceleration of the robot as the power setting is changed.

\sect{System design}

The robot moves to the end of the start tape and stops.

The timer event \bu{timer1} is set to occur every $t$ seconds. When it
occurs, the power of the robot is increased by 100. Therefore, the robot
runs at each power setting 100, 200, 300, 400, 500 for $t$ seconds. When
it has run at power 500 for $t$ seconds, the robot stops. The
measurements are explained below.

\sect{Constants}

\begin{itemize}
\item \p{MOTOR}: The initial power setting of the motors.

\item \p{ACCEL}: The time between occurrences of the event \bu{timer1}.  
\end{itemize}

\sect{Event handlers and subroutines}

\begin{itemize}

\item Subroutine \p{end\_of\_start\_found} is modified to set
\p{motor} to 0 and stop the robot.

\item Subroutine \p{drive\_straight} is modified to remove the condition
of stopping when the end tape is reached.

\item When the \bu{timer1} event occurs and the state is 3
(\bu{straight}):
\begin{itemize}
\item If \p{motor} is less than 500, increase the value of \p{motor} by 100.
\item If \p{motor} is equal to 500, call subroutine \p{stop}.
\end{itemize}


\end{itemize}

\sect{Programming notes}

{\raggedleft \hfill Program file \bu{speed3.aesl}}

\begin{itemize}

\item The counter \p{time} should be initialized to 0 in the
handler for the \bu{timer1} event when \p{motor} is 0. This will be more
accurate since the robot starts to move when this event handler is run
and not when the robot reaches the end of the start tape.

\item Initialization of \p{motor} to \p{MOTOR} should occur when the
center button event occurs so that you can run the program multiple
times without reloading.

\end{itemize}

\sect{Measuring the acceleration} Set the value of \p{timer1}
successively to three values (500 ms = 0.5 second, 1000 ms = 1 second,
2000 ms = 2 seconds) and run the program. Use a tape measure to measure
the distance that the robot moves and note the time as recorded in the
counter \p{time}.

A body that accelerates from rest travels a distance $s =
\frac{1}{2}at^2$, where $a$ is the acceleration. Therefore, we can
calculate the (average) acceleration as $a$ = $2s/t^2$.

For the values of the timer 0.5, 1, 2 seconds, I measured that the robot
traveled 25, 50, 100 centimeters in 2.5, 5, 10 seconds, respectively.
The accelerations are 8, 4, 2 cm/sec$^2$, showing that the faster you
``press the gas pedal down to the floor,'' the faster the robot
accelerates, and that the acceleration is inversely proportional to the
time it takes to press the pedal to the floor.

\sect{Experiments}

\begin{enumerate}

\item Experiment to see how consistent the measurements are.
Perform the speed measurement at the five power settings several times.

\item How sensitive is the speed measurement to the experimental setup.
What result do you get if you measure the speed over a 2-meter course.
Try making the top and bottom tapes closer together or farther apart and
see if that affects the results.

\item Experiment with the effect of friction on the speed of the robot.
Measure the speed on a bare tile or concrete floor, on a wooden desk, on
a sheet of plastic, and outside on the ground.

\item Making corrections to keep the robot within the rectangular frame
can cause inaccuracy in the measurement of the speed. Instead, you can
calibrate the motors so that the robot runs straight and then measure
the speed from on small strip of tape to another place one meter away.
Make small adjustments in the powers of the left and right motors until
the robot runs straight. Alternatively, use the calibration procedure
in the Thymio robot as explained \href{https://www.thymio.org/en:thymiomotorcalibration}{here}.

\item Modify the third program so that the acceleration is smooth.

\item Experiment with \emph{deceleration} where you start the robot at a
power setting of 500 and slowly reduce the power.

\end{enumerate}
