\chap{Line Following}\label{ch.line}

\sect{Specification}

Write a program that causes the robot to follow a line on the floor.

\sect{System design}

If the floor is light-colored, the ground proximity sensors will detect
a lot of reflected light, so we need a dark line that will reflect very little
light. This is easy to do by placing black electrician's tape on the
floor:

\gr{blacktape}{.8}

The tape must be wide enough (at least 4 cm) so that both ground sensors will
sense black when the robot is successfully following the
tape.\footnote{You can download and print out two closed black-on-white
lines from:\\
\url{https://aseba.wikidot.com/en:thymiobehaviourinvestigator}.\\ If
your floor is dark-colored, use white tape and modify the program
accordingly.} When the robot strays off the line, the direction of the
turn can be determined by which sensor measures a low level of light
(over the tape) and which measures a high level of light (over the
floor):

%\begin{figure}
\begin{center}
\begin{picture}(200,100)
%\put(0,0){\framebox(200,100)}
\thicklines
\put(0,20){\line(1,0){200}}
\put(0,70){\line(1,0){200}}
\thinlines
\put(80,0){
\put(0,0){\line(1,1){50}}
\put(0,0){\line(-1,1){50}}
\put(50,50){\line(-1,1){50}}
\put(-50,50){\line(1,1){50}}
\put(20,55){\framebox(10,10){}}
\put(5,72){\framebox(10,10){}}
\put(-5,95){\line(1,0){10}}
\put(45,45){\line(0,1){10}}
\put(80, 0){\makebox(40,10)[l]{\bu{Floor}}}
\put(80, 40){\makebox(40,10)[l]{\bu{Tape}}}
\put(80, 80){\makebox(40,10)[l]{\bu{Floor}}}
\put(25,75){\vector(1,1){20}}
}
\end{picture}
%\caption{The left sensor is off the tape and the right sensor is on the tape}
%\label{fig.one-off}
\end{center}
%\end{figure}

Once we know this direction we can turn the robot back onto the tape by
increasing the power of one motor and decreasing the power of the other
motor. In the drawing above, to cause the robot to turn right back onto
the tape, we cause the left motor to run forward at high power and the
right motor to run forward at low power, or even backwards to achieve a
tighter turn.

The difficult part of a program for line following is handling the case
when \emph{both} sensors no longer detect the line. The solution is to
remember the direction the robot was going when the first sensor detects
the floor. However, the sampling rate of the proximity sensors (10 times
a second) is not fast enough to do this if the robot is traveling at a
high speed. Therefore, a timer event is added and set to occur more
often, say, 50 times a second (every 20 milliseconds). When the event
occurs, the computation of the direction is performed as explained
below.

The direction of the turn is indicated by different colors of the top
LEDs.

\sect{State machine}

There are two states: 0 for \p{off} and 1 for \p{on}.

\sect{Constants}

\begin{itemize}
\item \p{THRESHOLD}: The threshold below which a sensor decides that it
is on a tape.

\item \p{DIFF}: The difference in sensor values above which
the direction is remembered.

\item \p{CHANGE}: The percentage change in the motor power when
turning.

\item \p{TIMER}: The period of the timer that computes the last
direction.

\item \p{INCREMENT} is the amount by which the motor power is increased
or decreased when the right and left buttons are pressed.
\end{itemize}

\sect{Variables}

\begin{itemize}
\item \p{state}: The state: \p{off} = 0 or \p{on} = 1. 
\item \p{motor}: The motor power setting.
\item \p{motor\_change}: The change to the motor power setting.
\item \p{diff}: The difference between the values of the left and right
ground sensors. 
\item \p{dir}: The direction the robot was turning when it left the tape
(1 = right, $-$1 = left).
\end{itemize}

\sect{Event handlers and subroutines}

\begin{itemize}

\item Setting the motor power: the handlers for the events
\bu{button.left} and \bu{button.right} and the subroutine
\p{set\_circle\_leds} were described in Chapter~\ref{ch.fast}.

\item When the center-button event occurs: if the button is released,
the state and the motors are changed from on to off and from off to on.

\item When a proximity event occurs, the subroutine \p{drive} is called.
\begin{itemize}
\item If both sensors detect the tape, the robot drives straight.
\item If only one of the sensors detects the tape, the robot turns in
the correct direction to return to the tape.
\item If neither sensor detects the tape, the robot turns in the
direction indicated by the variable \p{dir}.
\end{itemize}

\item When the \bu{timer0} event occurs: The range of values of the
ground sensors is very large---between 0 and 1023---so the two
sensors will almost certainly measure slightly different values even if
both are above the tape or above the floor. Therefore, the direction is
computed only if the absolute value of the difference between the
measured values is significant---above a threshold determined by the
constant \p{DIFF}. If so, the variable \p{dir} is set to 1 if the value
of the left sensor is greater than that of the right sensor and, if not,
it is set to $-$1.

\end{itemize}

\newpage

\sect{Programming notes}

The operator for absolute value is defined in AESL. The
outline of the \bu{timer0} event is:
\begin{verbatim}
  onevent timer0
    diff = prox.ground.delta[0] - prox.ground.delta[1]
    if  abs diff >= DIFF then
      if  diff > 0 then dir = 1 else dir = -1 end
    end
  \end{verbatim}

{\raggedleft \hfill Program file \bu{line.aesl}}

\sect{Experiments}

\begin{enumerate}

\item What is the maximum speed at which the robot can reliably
following the line? (Here, \emph{reliably} means that the robot can
successfully follow the line if you run the program several times.) The
faster the robot, the more likely it is that the robot will lose the
line and not be able to return to it.

\item Experiment with the power setting of the turns. My program sets
one motor to move forward and one to move backwards. This turns the
robot rapidly in the correct direction but slows it down. A gentle turn
may or may not be better.

\item What is the largest angle that the robot can follow? Can it follow
a tape that has a full ($90^\circ$) left or right turn?

\item Modify the program so that instead of using a single value for the
detection threshold, there are separate thresholds $t_w$ and $t_b$ such
that the robot is considered to be on the tape if the sensor value is
less than $t_b$ and off the tape if the sensor value is greater than
$t_w$.\footnote{This experiment and the following one are based on the
 line-following program from:\\
\url{https://aseba.wikidot.com/en:thymiobehaviourinvestigator}.}

\item Modify the program so that the robot calibrates the
sensors---determines the detection thresholds---by itself. Place the
robot with both sensors over the tape and press a button; the program
will calculate $t_b$. Now place the robot with both sensors over the
floor and press another button; the program will calculate $t_w$.

\end{enumerate}
