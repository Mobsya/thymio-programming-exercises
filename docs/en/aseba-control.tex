\chapter{Controlling the Robot}\label{ch.control}

All our projects contain an algorithm that \emph{decides} what to do.
The proximity sensors measure a value and if it is above or below a
threshold, the program decides (using an \p{if}-statement) whether to
start or stop the robot, or to turn it in one direction or another. These
algorithms are called \emph{control algorithms} and there is a
sophisticated mathematical theory of control that is fundamental in
robotics. In this chapter, we develop several projects that demonstrate
various control algorithms.

The projects concern searching for and approaching an object, as was
done in the cat-catches-mouse project in Chapter~\ref{ch.cat}. We
simplify the projects in order to the focus on the control algorithms.
The presentation of the projects will be more informal than in
previous chapters on the assumption that you now have enough experience
to design state machines, declare constants and variables, and write
event handlers and subroutines.

I built an object from a lipstick container wrapped in aluminum foil.
The object must be narrow so that an individual horizontal sensor
will be able to locate it accurately.

\gr{control}{.6}

Before you proceed with the projects, experiment to find out the range
of values returned by the horizontal sensors as the object
gets closer. Recall that you can examine the values in the
\bu{Variables} pane of Aseba after checking the box labeled \p{auto}.
Define a constant \p{THRESHOLD} to be close to the highest value. I
found that the values ranged between 0 and 4100 or a bit more, so I
defined \p{THRESHOLD} as 4000.

The first part of the chapter described two projects for searching for
the object, one using a simple on-off controller and the other using a
proportional controller. The second part describes three projects for
approaching the object, using an on-off and a proportional controller,
as well as a more sophisticated \emph{proportional-integral} controller.

\centeredbox{Tip: You can follow the changes in the motor power by
examining the variables table as described on
page~\pageref{p.variables}. A better way is to program the circle LEDs
to display the motor power as described on page~\pageref{p.leds}.}

\section{Searching}

\sect{An on-off controller}

\textbf{Specification}: When the leftmost horizontal sensor of the
Thymio detects the object, it turns rapidly to the right until the
rightmost sensor detects the object and then stops.\footnote{In all
the projects, the touching the center button starts and stops the
program and motors.}

{\raggedleft \hfill Program file \bu{scan-on-off.aesl}}

The first control algorithm is very simple. When the value of the
leftmost sensor is greater than the threshold turn the motors
on to their highest positive and negative values (500 and $-$500). When
the value of the rightmost sensor is greater than the threshold, turn
the motors off:

\begin{verbatim}
  onevent prox
    if  prox.horizontal[0] > THRESHOLD then
      motor.left.target  = -MOTOR 
      motor.right.target = MOTOR
    elseif  prox.horizontal[4] > THRESHOLD then
      motor.left.target  = 0
      motor.right.target = 0
    end
\end{verbatim}

This is called the \emph{on-off} algorithm, because the motors are
either on at full power or off.\footnote{Another, colorful, name is the
\emph{bang-bang} algorithm!} Run the program. What happens? When I ran
the program, the robot stopped with the object \emph{past} the
rightmost sensor. When the robot is moving at full speed, the rate at
which the \p{prox} event occurs is simply not fast enough to stop the
robot in time. As you know from riding in a car, starting and stopping
as fast as possible is very uncomfortable. Fast acceleration and braking
is also bad for the engine and brakes. For these reasons, the on-off
algorithm is not used in practice.

\sect{A proportional controller}

To develop an algorithm that is better than the on-off algorithm, we
take inspiration from riding a bicycle. Suppose that you are riding your
bicycle and see that the traffic light ahead has turned red. You
\emph{don't} wait until the last moment and then squeeze hard on the
brake lever; if you did so, you might be thrown from the bicycle! What
you do is to slow your speed gradually: First, you stop pedaling; then,
you squeeze the brake gently to slow down a bit more; finally, when you
are at the light and going slowly, you squeeze harder to fully stop the
bicycle.

The algorithm used by a bicycle rider can be expressed as: to
stop gradually, reduce your speed more as you get closer to the
object. The decrease in speed is
\emph{proportional} to how close you are to the light: the closer you
are, the more you slow down.

{\raggedleft \hfill Program file \bu{scan-p.aesl}}

To implement a proportional algorithm we have to \emph{measure} how
close the object is from rightmost sensor. The language of the
following description might seem strange but the intent should be clear:

\begin{quote}
When detected by the leftmost sensor, the object is 0 units close;\\
When detected by the next sensor to the right, it is 1 unit close;\\
When detected by the center sensor, it is 2 units close;\\
When detected by the next sensor to the right, it is 3 units close;\\
When detected by the rightmost sensor, the robot is at the object.
\end{quote}

The algorithm reduces the motor power setting in proportion to the
number of units that are measured. Define a constant \p{DELTA} for the
amount by which the power is reduced for each unit. When the object is
detected by the $n$'th sensor ($n$ = 0 .. 3), the power is reduced by
$n$ times \p{DELTA}. When the 4th sensor detects the object, the robot stops.
The event handler is shown in
Figure~\ref{f.pro}.

\begin{figure}
\begin{verbatim}
  onevent prox
    if  prox.horizontal[4] > THRESHOLD then
      motor.left.target  = 0
      motor.right.target = 0
    elseif  prox.horizontal[3] > THRESHOLD then
      motor.left.target  = -MOTOR + DELTA * 3
      motor.right.target =  MOTOR - DELTA * 3
    elseif  prox.horizontal[2] > THRESHOLD then
      motor.left.target  = -MOTOR + DELTA * 2
      motor.right.target =  MOTOR - DELTA * 2
    elseif  prox.horizontal[1] > THRESHOLD then
      motor.left.target  = -MOTOR + DELTA * 1
      motor.right.target =  MOTOR - DELTA * 1
    elseif  prox.horizontal[0] > THRESHOLD then
      motor.left.target  = -MOTOR + DELTA * 0
      motor.right.target =  MOTOR - DELTA * 0
    end
\end{verbatim}
\caption{Searching with a proportional algorithm}\label{f.pro}
\end{figure}

Run the program, experimenting with the values for \p{DELTA},
so that the robot starts turning very fast but then gently
slows down, just as if you were applying more and more pressure to the
brake lever on the bicycle. The robot should stop with the rightmost
sensor precisely opposite the object.

\textbf{Programming note}: Why are the sensors checked in the order 4,
3, 2, 1, 0 and not in the order 0, 1, 2, 3, 4? The answer is in the
footnote, but think about it before you look.\footnote{If the object
is detected by \emph{two} sensors at the same time, the lower power
setting of the sensor \emph{closer to} the rightmost sensor should be
used. This way the slowing down of the robot is smoother. Experiment
with the other order and see what happens.}


\section{Approaching}

In this section we develop algorithms for slowing down and stopping the
robot. It is similar to the \bu{pounce} behavior of the
cat-catches-mouse project in Chapter~\ref{ch.cat}. The proportional
algorithm in the previous section used a rather arbitrary measure of
closeness: the number of sensors between the current position of the
object and the rightmost sensor. In this section, the power setting of
the motors will be proportional to a measure of the distance between the
sensor and the object.

\textbf{Specification}: The object is placed in front of the center
horizontal sensor. The robot moves towards the object and
stops just before hitting it.

Previously, you were asked to determine the upper limit of the value of
the horizontal sensor when it detected the object. For the
projects in this section, a value close to the upper limit is
given by the constant \p{TARGET}. Instead of checking if the value of
the sensor is above a threshold, we compute the \emph{error} defined as:
\begin{displaymath}
\mathit{error} = \mathit{target\ distance} - \mathit{measured\ distance} 
\end{displaymath}
The robot stops when the error is below a value given by the constant
\p{TARGET\_ERROR}.

\sect{An on-off controller}\label{s.on-off}

{\raggedleft \hfill Program file \bu{approach-on-off.aesl}}

Here is the event handler for an on-off controller. It turns the motors
off when the error is small, but leaves them at full power when the
error is large:

\begin{verbatim}
  onevent prox
    error = abs (TARGET - prox.horizontal[2])
    if error < TARGET_ERROR then
      callsub stop
    else
      motor.left.target  =  MOTOR
      motor.right.target =  MOTOR
    end
\end{verbatim}

\textbf{Programming note}: The absolute value is used because the value
of the sensor can be slightly higher than \p{TARGET} so the error can be
negative. What problem can this cause? The answer is in the footnote,
but think about it before you look.\footnote{If the value of the sensor
is large, the absolute value of the difference could be larger than
\p{TARGET\_ERROR} and the robot will continue moving instead of
stopping. Modify the program to solve this problem.}

When I ran this program the robot moved so fast that it collided with
the object!

\sect{A proportional controller}

{\raggedleft \hfill Program file \bu{approach-p.aesl}}

In a proportional controller the motor power is proportional to the
error: The larger the error---the farther the robot is from the
object---the higher the power; as the robot approaches the object,
the error becomes small and the power is reduced. The power is
determined by multiplying the error by a number called the \emph{gain}
for which we define a constant \p{GAIN\_P}. Be careful to avoid
saturation: the power must never be higher than the highest allowed
value, which is 500 for the Thymio robot:

\begin{verbatim}
    motor_power = error * GAIN_P
    if motor_power > 500 then motor_power = 500 end
\end{verbatim}

Experiment to find a value for the gain that gives good results. I was
able to get the robot to stop just before hitting the object.

\textbf{Programming note}: The variable \p{error} has values in the
range from 0 to about 4000, while we want the motor power to have values
in the range 0 to 500. Therefore, the gain should be $1/8 = 0.125$ (or
less). In Section~\ref{s.measure}, we solved this problem by scaling
the integer division, but there is a better solution.

The Aseba native function \p{math.muldiv(A, B, C, D)} computes $A = (B
\cdot C) / D$ in 32-bit arithmetic. Therefore, \p{GAIN\_P} can be
defined to be 8 and the value of \p{motor\_power} computed without using
a scaling factor:

\begin{verbatim}
  error = abs (TARGET - prox.horizontal[2])
  call math.muldiv(motor_power, error, 1, GAIN_P)
  if motor_power > 500 then motor_power = 500 end
\end{verbatim}

\textbf{Programming note}: In my program, the statements in the event
handler for \p{prox} are only run if the state is \p{on} and not \p{off}.
However, I placed the computations of \p{error} and \p{motor\_power}
before checking the state so that I can observe them in the
\p{Variables} pane without actually running the algorithm.

\sect{A proportional-integral controller (advanced)}

{\raggedleft \hfill Program file \bu{approach-pi.aesl}}

While the proportional controller does not collide with the object, it
may not get as close as we want it to. A \emph{proportional-integral
controller} can to a better job by using the accumulated error to
control the power setting. In mathematics, an accumulated sum is called
an integral.

To implement a proportional-integral controller, use a variable
\p{error\_sum} to accumulate the error and then add it to the motor
power after scaling by the reciprocal of the integral gain \p{GAIN\_I}.
To prevent \p{error\_sum} from growing too large, the \p{error} is
divided by a scale factor \p{SUM\_FACTOR} before it is added to
\p{error\_sum}:

\begin{verbatim}
  error = abs (TARGET - prox.horizontal[2])
  error_sum = error_sum + (error / SUM_FACTOR)
  call math.muldiv(motor_power, error, 1, GAIN_P)
  motor_power = motor_power + error_sum / GAIN_I
  if motor_power > 500 then motor_power = 500 end
\end{verbatim}

Experiment with the gains to see if you can detect a difference between
the proportional controller and the proportional-integral controller.
