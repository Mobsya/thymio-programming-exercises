% !TeX root = aseba.tex

\chapter{Nonlinear Sensors}\label{ch.nonlinear}

In Chapter~\ref{ch.fast} we measured the speed of the Thymio robot for
various settings of the motor power and found that the speed was a
linear function of the power $v \approx 0.03p$. For example, a power
setting of $300$ gave a speed of about $9$ cm/sec. In this chapter we
will look at the values returned by the horizontal proximity sensors and
ask: Are the values a linear function of the distance to an obstacle?

\section{Measuring the horizontal proximity sensors}

\sect{Specification}

Construct a graph showing the values returned by the center horizontal
sensor for an obstacle placed at 1 cm intervals in front of
the robot.

\sect{System design}

Tape a ruler on your table and carefully place the robot so that its
front is at the 0 mark of the ruler. Place an obstacle exactly at the 1
cm mark on the ruler and touch the center button. Store the value of
the center proximity sensor. Repeat for 2 cm, 3 cm, \ldots, until the
value goes to zero.

The value returned by the sensor depends not only on the distance of the
obstacle but also on how precisely it is centered in front of the
obstacle. For each distance, move the obstacle slightly left and right
until the value is as large as possible.\footnote{The values change all
the time so you won't be able to get the true maximum.} You can read the
value of the sensor in the \bu{Variables} pane at the left of the Aseba
Studio window; click the check box \bu{auto} so that the display will be
dynamically refreshed.

\sect{Variables}

\begin{itemize}

\item \p{proximity}: An array for storing the values read from the
sensor. Eight elements should be sufficient: by experiment, I found that
the sensor is unlikely to detect an obstacle further than 8 cm away.
Initialize each element of the array to zero so that you can determine
which values have been entered from a new run of the program.

\item \p{count}: An index for the array. 

\end{itemize}

\sect{Event handlers and subroutines}

When the center button is released, the current value of the center
horizontal sensor is stored in the first unused element of the
array \p{promixity}.

\sect{Programming notes}

{\raggedleft \hfill Program file \bu{nonlinear1.aesl}}

Aseba can determine the size of an array from the number of elements in
an initial value.

\begin{verbatim}
  var proximity[] = [0,0,0,0,0,0,0]
\end{verbatim}

\sect{Running the program}

Run the program and touch the center button for each position of the
obstacle. The values in the array \p{proximity} will be displayed in the
\bu{Variables} pane. Construct a graph with distance on the x-axis and
the sensor values on the y-axis. Better, run the program several times
and use the average values of the sensors for each distance. Averaging
values and drawing the graph is easy if you have spreadsheet software on
your computer.\footnote{You need not copy the values by hand from Aseba
to the spreadsheet. After running the program, select \bu{Files/Export
memory content} and enter a file name such as \bu{data.txt}. Open the
file \bu{data.txt} in an editor and look for the line
\p{thymio.proximity}; the contents of the array are listed on this
line and you can copy and paste them into your spreadsheet.}

Figure~\ref{fig.nonlinear} shows the results of two runs and their
average. The function is reasonably linear in the range 3--5 cm, but
nonlinear outside that range.


\begin{figure}
\gr{horiz}{.6}
\caption{Sensor value for distances 1--7 cm}\label{fig.nonlinear}
\end{figure}

\sect{Experiments}

\begin{enumerate}
\item Modify the program so that it stores the results of three
sequences of measurements and computes the average.

\item Do the values measured by sensor depend on the obstacle?
Experiment with obstacles of different shapes and materials.

\item Modify the program so that it records the values of the other four
horizontal sensors. Do the different sensors return similar values?
 
\end{enumerate}

\section{Calibrating the horizontal proximity sensors}

Nonlinear sensors can be used if they are \emph{calibrated}.
The results of an experimental measurement (as described in the previous
section) can be used to translate the raw sensor data into actual
distances.

\sect{Specification}

When an obstacle is placed in front of the center horizontal sensor and
the center button touched, the circle LEDs on the top of the robot
display the distance from the robot.

\sect{System design}

An array stores the values measured for each distance. When the button
is released, the value of the sensor is compared with the elements of
the array. The first index at which the element of the array is lower
than the measured value is taken to be the distance, and that number of
LEDs are turned on. For example, let the values in the array be:

\begin{verbatim}
var proximity[] = [4100, 4045, 3650, 2705, 1850, 1245, 965, -1]
\end{verbatim} 

If the sensor measures a value of 3000, then \p{proximity[3]} (2705) is
less than 3000, so the loop stops with the value 3 in \p{distance}.
If the sensor measures 2000, the distance is 4 cm.

\sect{Variables}

\begin{itemize}

\item \p{proximity}: The array stores the values of the sensors
previously measured.

\item \p{distance}: The array index; its final value is the distance to
the obstacle.

\end{itemize}

\sect{Event handlers and subroutines}

\begin{itemize}

\item When the center button is released, a loop searches the array
\p{proximity} comparing the values with the value measured by the
sensor.

\item \p{set\_circle\_leds}: Light the number of LEDs given by the
variable \p{distance}; it is similar to the subroutine in
Chapter~\ref{ch.fast} that displayed the motor setting.

\end{itemize}

\sect{Programming notes}

{\raggedleft \hfill Program file \bu{nonlinear2.aesl}}

We used a well-known programming trick called searching an array with a
\emph{sentinel}. The last element of the array is given the value $-1$,
which is less than $0$, the \emph{smallest} possible value of the
sensor. This ensures that the loop will \emph{always} terminate.

\sect{Running the program}

When I ran the program, the results were accurate for distances in the
range 1--5, but less accurate for larger distances.

\sect{Experiments}

\begin{enumerate}

\item The transition between $n$ cm and $n+1$ cm is taken when the
obstacle is bit over $n$ cm from the robot. Modify the program so that
the transition is taken halfway between the two distances.

\item Modify the program so that it can measure distances of half a
centimeter. One solution is to extend the calibration to sample values
for every one-half centimeter. Alternatively, you could try
\emph{interpolation}, where the transition value for a half centimeter
distance are taken as the average of the two neighboring values.

\end{enumerate}
