% !TeX root = aseba.tex

\chapter{Advanced Odometry}\label{ch.odo-advanced}

In the previous chapter, the distance traveled in one dimension was measured.
For a robot like the Thymio that can turn,
we want to track its position in the plane in two dimensions
$x$ and $y$, as well as its \emph{heading}, the direction in which
the robot is pointing.

This chapter presents the basic concepts of odometry for a turning robot.
Since Aseba only supports 16-bit integer computation,
while the trigonometric computations that are needed
are usually done with real numbers,
scaling factors must be used and it
will be rather difficult to keep track of them.

\sect{Specification}

The Thymio robot moves a short distance with the right wheel moving
faster than the left wheel. Compute the new position and heading of the robot.

\sect{Design}

Consider a robot that moves a short distance at an angle:

\begin{center}
\begin{picture}(300,120)
%\put(0,0){\framebox(300,120){}}
\put(0,20){
\drawline(0,0)(300,0)
\drawline(0,0)(280,100)
\dashline{5}(300,0)(280,100)
\dashline{5}(260,0)(240,85)
\dashline{5}(220,0)(200,70)
\put(40,5){\makebox(10,10){$\theta$}}
\put(190,40){\makebox(20,10){$d_l$}}
\put(230,40){\makebox(20,10){$d_c$}}
\put(270,40){\makebox(20,10){$d_r$}}
\put(255,-5){\rule{10pt}{10pt}}
\put(220,0){\circle*{10}}
\put(300,0){\circle*{10}}
}
\put(240,0){\makebox(40,10){\textit{baseline}}}
\put(280,5){\vector(1,0){20}}
\put(240,5){\vector(-1,0){20}}
\end{picture}
\end{center}
The square indicates the center of the robot and the circles indicate
the left and right wheels. The distance between them is \textit{baseline} and
$d_l, d_r, d_c$ represent the distances moved by the two wheels and the center
of the robot.

The circumference of a circle of radius $r$ is given by $2\pi r$ and in
general the length of an arc of angle $\theta$ (in radians)\footnote{A
radian is a unit used to measure angles. A circle of $360^\circ$ is
defined to be $2\pi\approx 6.28$ radians. It follows that $180^\circ =
\pi$ radians, $90^\circ = \pi/2$ radians and so on.} is given by $\theta
r$. Therefore, we have:

\begin{displaymath}
\theta = d_l/r_l = d_r/r_r = d_c/r_c,
\end{displaymath}
where $r_l, r_r, r_c$ are the distances of the wheels and the center from the origin of the turn.

The angle $\theta$ can be computed from the distances traveled and the
baseline:
\begin{eqnarray*}
r_r\cdot\theta &=& d_r,\\
r_l\cdot\theta &=& d_l,\\
r_r \cdot\theta - r_l \cdot\theta &=& d_r - d_l,\\
\theta &=& (d_r - d_l) / (r_r - r_l),\\
\theta &=& (d_r - d_l) / \textit{baseline}.\\
\end{eqnarray*}

We assume that the center of the robot is halfway between its wheels,
so $r_c =(r_l+r_r)/2$,
and therefore $r_c = d_c / \theta =(d_l / \theta + d_r / \theta)/2$; multiplying
by $\theta$ gives $d_c =(d_l+d_r)/2$.


If the distance moved is small, the line labeled $d_c$ is approximately
perpendicular to the radius through the final position of the robot:
\begin{center}
\begin{picture}(300,100)
%\put(0,0){\framebox(300,100){}}
\drawline(0,0)(260,0)
\drawline(0,0)(200,100)
\dashline{5}(260,0)(200,100)
\dashline{5}(200,0)(200,100)
\dashline{5}(260,0)(260,100)
\put(40,5){\makebox(10,10){$\theta$}}
\put(230,40){\makebox(20,10){$d_c$}}
\put(255,-5){\rule{10pt}{10pt}}
\put(200,70){\makebox(10,10){$\theta$}}
\put(250,15){\makebox(10,10){$\theta$}}
\put(220,5){\makebox(10,10){\textit{dx}}}
\put(185,40){\makebox(10,10){\textit{dy}}}

\put(285,20){\vector(-1,0){20}}
\put(290,13){\makebox(10,10){
  \shortstack[l]{
    \textit{change in}\\
    \textit{heading}
  }}}

% Right triangle indications
\drawline(197,98)(199,95)
\drawline(199,95)(202,97)
\drawline(200,5)(205,5)
\drawline(205,5)(205,0)
\end{picture}
\end{center}
By similar triangles, we see that $\theta$ is the change in heading of the robot.

By trigonometry, we have that:
\begin{eqnarray*}
\textit{dx} &=& - d_c \cdot \sin \theta,\\
\textit{dy} &=& d_c \cdot \cos \theta.\\
\end{eqnarray*}
These formulas show how to compute the changes \textit{dx}, \textit{dy} and
$\theta$ when the robot moves a short distance.
To implement odometry,
the changes are computed at frequent intervals
and the actual position and heading
are updated with these changes.
For simplicity and given the limitations of the measurements of $d_l$ and $d_r$,
we will implement the computation of
a single change over a relatively long period of time.

\sect{Constants}

\begin{itemize}
\item \p{TIME}: Duration of the run (30 tenths of a second)
\item \p{BASELINE}: The distance between the left and right wheels (95 mm)
\item \p{LEFT}: Power setting of left wheel (200)
\item \p{RIGHT}: Power setting of right wheel (300)
\item \p{SPEED}: The speed of robot per 100 power setting (32 mm/sec)
\end{itemize}

\sect{Programming}

{\raggedleft \hfill Program file \bu{odo3.aesl}}

It would be easy if we could ask for a single timer event to occur
after three seconds, but the Thymio timers cause an event to occur
immediately when the program runs.
Therefore, the timer is set to expire every tenth of a second and the
odometry computation is done when 30 timer events have occurred.

The distances traveled by the two wheels are computed from the power settings:
\begin{verbatim}
  call math.muldiv(dleft, motor.left.speed, SPEED*TIME, 100*10*10)
  call math.muldiv(dright, motor.right.speed, SPEED*TIME, 100*10*10)
  dcenter = (dleft+dright)/2
\end{verbatim}
The result is divided by a factor of 100 for the motor powers,
10 for the power to speed conversion (3.2 cm/sec) and again by 10
because the time is tenths of a second.

The change of heading is computed from the distances traveled by the two wheels
and the baseline:
\begin{verbatim}
  theta = ((dright-dleft)*10*100)/BASELINE
\end{verbatim}
The factor 10 cancels out the factor of 10 in the baseline that
is given in millimeters.
The factor 100 gives the result in hundreths of radians,
transforming the real range 0--3.14 to the integer range 0--314.
(For simplicity we consider only positive angles.)

The sine and cosine are computed:
\begin{verbatim}
  call math.cos(tcos, theta*100)
  call math.sin(tsin, theta*100)
\end{verbatim}
In Aseba, the parameter to the trigonometric functions is within
the entire 16-bit range -32768--32767, which represents the range
from $-\pi$ to $\pi$ radians.
Multiplying the angle by 100 will cause the range of positive
angles 0--314 to become 0--31400, approximately the range
of the positive angles as 16-bit integers.

We can now compute the change in the x- and y-positions:
\begin{verbatim}
  call math.muldiv(dx, dcenter, -tsin, 32767)
  call math.muldiv(dy, dcenter, tcos, 32767)
\end{verbatim}
The result is divided by 32767 because the result of the trigonometric
functions in Aseba is within the entire 16-bit range -32768--32767,
instead of the normal range of real numbers $-1.0$--$1.0$.

\sect{Running the program}

Running the program gave the following results which are reasonable:
\begin{center}
\begin{tabular}{|l|c|}
\hline
\p{dleft} & 20\\\hline
\p{dright} & 29\\\hline
\p{dcenter} & 24\\\hline
\p{theta} & .94\\\hline
\p{dx} & -18\\\hline
\p{dy} & 15\\
\hline
\end{tabular}
\end{center}
The units are centimeters except for $\theta$ which is in radians
(.94 radians $\approx 54^\circ{}$).

Running the program several times gives different results because the values
returned by \p{motor.X.speed} are not consistent.

\sect{Experiments}

\begin{itemize}
\item Run the program for different periods of time.
How does this affect the accuracy and consistency of the odometry computation?
\item The Thymio robot has a hole that can be used to hold a marker.
Mark the actual path of the robot and measure $d_c$.
Can you use this to compute $d_l$ and $d_r$?
If so, perform the odometry computation with these values.
Is the computation more accurate and consistent than the computation
using the measurements of \p{motor.X.speed}?
\end{itemize}
