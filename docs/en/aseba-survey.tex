\chap{Thymio the Surveyor}\label{ch.survey}

\emph{Land surveying} is a set of techniques for determining the precise
position of points on the earth.\footnote{See the
\href{https://en.wikipedia.org/wiki/Surveying}{Wikipedia article on
surveying} for an overview.} Since ancient times, surveying was used to
mark out fields and to build large structures. This project uses the
Thymio to determine the location of an object using two techniques:
measuring the angle and distance to the object, and measuring the angle
to the object from two different positions. The results will be very
inaccurate, but the techniques and computations reflect actual practice.

Take a piece of paper marked with a grid of 1 cm squares. Place the
Thymio so that its front panel is at the left edge of the grid and place
an object somewhere on the grid:\footnote{Print the file \bu{grid.pdf}
in the \p{images} directory. The file was generated from the \LaTeX\
file \bu{grid.tex}. We used reflecting tape on the object as described
on page~\pageref{p.reflect} to increase the detection range.}

\gr{survey}{.6}

\sect{Measuring angles using the Thymio}

The sensors are on the front panel of the robot, which forms an arc of a
circle whose center is the hole used to place a pencil for drawing.
Place a protractor with its center over the hole and use a thread to form
a line from the center to a sensor (Figure~\ref{fig.angles}). Read the
angle from the protractor. Since the sensors are spaced uniformly, we
define a constant \p{THETA} and the five angles will be:

\begin{center}
\p{-2*THETA, -THETA, 0, -THETA, 2*THETA}.  
\end{center}

\begin{figure}
\gr{angle}{.5}
\caption{Measuring the angle between the sensors}\label{fig.angles}
\end{figure}

\section{Determining position from an angle and a distance}

Chapter~\ref{ch.nonlinear} describes how to determine the distance to an
object by relating the values returned by the horizontal proximity
sensors to distance in centimeters. The angle to the object is
determined by the horizontal sensor that returns the highest value.

\sect{Computing the coordinates}

Figure~\ref{fig.geom} shows the geometry of the surveyor. The object is
indicated by the dark point. The Thymio is at the origin of a coordinate
system with $(0,0)$ defined as the hole. The $x$-axis goes through the
axis of the wheels, while the $y$-axis points forwards. Given the angle
$\theta$ to the object as defined by the horizontal proximity sensor
with the highest value, and the distance $d$ to the object defined by the
value returned by that sensor, the coordinates can be computed using
elementary trigonometry.

\begin{figure}[htb]
\setlength{\unitlength}{1.4pt}
\begin{center}
\begin{picture}(185,100)
%\put(0,0){\framebox(185,100){}}
\put(0,30){
\put(0,0){\framebox(30,30){}}
\put(30,15){\oval(30,30)[r]}
\put(10,15){\circle{4}}
\put(10,15){\line(1,0){150}}
\put(10,-30){\line(0,1){90}}
\put(-5,60){\makebox(30,10){$x$-axis}}
\put(160,10){\makebox(30,10){$y$-axis}}

\put(10,15){\vector(2,1){110}}
\put(120,15){\line(0,1){55}}
\put(45,20){\makebox(10,10){$\theta$}}
\put(60,45){\makebox(10,10){$d$}}
\put(130,35){\makebox(20,10){$x=d \sin \theta$}}
\put(65,5){\makebox(20,10){$y=d \cos \theta$}}
\put(121,71){\circle*{4}}
}
\end{picture}
\setlength{\unitlength}{1pt}
\caption{The geometry of the surveyor}\label{fig.geom}
\end{center}
\end{figure}

\sect{Variables and constants}

\begin{itemize}

\item \p{THETA}: The angle between two sensors.

\item \p{OFFSET}: The distance between the hole in the Thymio and its
front panel.

\item \p{proximity}: An array for storing the values read from the
sensor.

\item \p{max\_prox}: The proximity sensor with the maximum value.

\item \p{max\_value}: The value returned by sensor \p{max\_prox}.

\item \p{distance}: The distance from the Thymio hole to the object.

\item \p{angle}: The angle from the $y$-axis to the object.

\item \p{x, y}: The coordinates of the object.

\end{itemize}

You may need additional variable for indices and for storing temporary
values.

\sect{Event handlers and subroutines}

It is sufficient to use one event like touching a button to initiate the
measurements and computations, but I used two events so that the
distance and angle are displayed on the top, bottom and circle LEDs
before the computation is done, and later the coordinates are displayed.
The subroutines for displaying data on the LEDs will not be described
here; you are invited to design your own method for doing this.

There are three subroutines for performing the computations:

\begin{itemize}

\item \p{get\_angle}: Get the angle by checking which horizontal
proximity sensor returns the maximum value.

\item \p{get\_distance}: Compute the distance corresponding to the
maximum sensor value.

\item \p{get\_coordinates}: The computation of the coordinates from the
distance and angle is conceptually very simple, but difficult to do on
the Thymio because the trigonometric functions do not use standard
values for parameters like degrees or radians. Study the computations in
Chapter~\ref{ch.odo-advanced} to see how this is done.

\end{itemize}

\sect{Programming notes}

{\raggedleft \hfill Program file \bu{survey1.aesl}}

\sect{Experiments}

\begin{enumerate}

\item Make a more accurate determination of the angle. If two horizontal
sensors return values that are ``more or less'' the same, use the
average of their two angles.

\end{enumerate}

\section{Determining position by triangulation}

Surveyors use microwave and light waves to measure distance. Before
these were available, it was extremely difficult to accurately measure
distances, especially distances to very distant or inaccessible objects.
Angles were always relatively easy to measure. Therefore,
\emph{triangulation} was used in surveying. If you know two angles of a
triangle and the length of the included side, you can compute the
lengths of the other sides. Once these are known, the position of a
distant object can be computed as we did above.

\begin{figure}
\setlength{\unitlength}{1.4pt}
\begin{center}
\begin{picture}(130,130)
%\put(0,0){\framebox(130,130){}}
\multiput(0,0)(0,100){2}{
\put(0,0){\framebox(30,30){}}
\put(30,15){\oval(30,30)[r]}
\put(10,15){\circle{4}}
\put(10,15){\line(1,0){120}}
}
\put(10,15){\line(0,1){100}}
\put(130,75){\circle*{4}}
\put(10,15){\line(2,1){120}}
\put(10,115){\line(3,-1){120}}

\put(45,20){\makebox(10,10){$\alpha$}}
\put(45,104){\makebox(10,10){$\beta$}}
\put(110,70){\makebox(10,10){$\gamma$}}
\put(12,20){\makebox(10,10){$\alpha'$}}
\put(12,100){\makebox(10,10){$\beta'$}}
\put(60,45){\makebox(10,10){$b$}}
\put(60,85){\makebox(10,10){$a$}}
\put(10,65){\makebox(10,10){$c$}}

\end{picture}
\setlength{\unitlength}{1pt}
\caption{Triangulation}\label{fig.triangulation}
\end{center}
\end{figure}

Consider the setup shown in Figure~\ref{fig.triangulation}. The Thymio first
measures the angle $\alpha$; then it is moved a distance of $c$ and the
angle $\beta$ can be measured. The lengths $a$ and $b$ can be computed
using the \emph{law of sines} for a triangle:

\begin{displaymath}
\frac{a}{\sin\alpha'} = \frac{b}{\sin\beta'} = \frac{c}{\sin\gamma}.
\end{displaymath}

We have used the interior angles of the triangle $\alpha' = 90^{\circ} -
\alpha, \: \beta' = 90^{\circ} - \beta$.

Given the angle $\alpha$, let us we compute the length $b$. We have:

\begin{displaymath}
\gamma = 180^{\circ}-\alpha'-\beta' = 180^{\circ} - (90^{\circ} -
\alpha) - (90^{\circ} - \beta) = \alpha + \beta.
\end{displaymath}
From the law of sines:
\begin{displaymath}
b = \frac{c\sin\beta'}{\sin\gamma} =
\frac{c\sin (90^{\circ} - \beta)}{\sin (\alpha + \beta)} =
\frac{c\cos\beta}{\sin (\alpha + \beta)}.
\end{displaymath}

\sect{Programming notes}

Two events were used: touch the forwards button to measure $\alpha$,
then move the Thymio to the new position and touch the backwards button
to measure $\beta$ and compute $b$.

The object was placed at coordinates $(8,12)$ and the two measurements
were done after moving the Thymio 24 cm. The results were
$\alpha=20^{\circ}$ and $\beta=40^{\circ}$, giving $b=22$. Measuring the
distance from the center hole to the object showed that this is a
reasonable value. Of course, the extremely coarse measurement of the
angles makes an accurate determination of the position impossible.

{\raggedleft \hfill Program file \bu{survey2.aesl}}

\sect{Experiments}

\begin{enumerate}

\item Program the Thymio so that it moves to the second position by
itself after making the first measurement.

\end{enumerate}
