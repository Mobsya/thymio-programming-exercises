% !TeX root = aseba.tex

\chap{Introduction}\label{ch.intro}

This document describes projects for the Thymio robot using \emph{AESL},
the textual programming language of Aseba, and its \emph{Studio}
development environment. The document assumes that you are familiar with
the Thymio robot, for example, by constructing projects using VPL, the
visual programming language. You must understand the concept of an
\emph{event} which causes an \emph{action}. Elementary versions of some
of these projects are contained in the VPL tutorial; here we present
better solutions that use the capabilities of AESL.

\sect{Project structure}

Each chapter is structured as follows:

\begin{itemize}
\item Specification of the problem to be solved.
\item Design of the program.
\item Construction of a state machine for the robot.
\item Declarations of constants and variables.
\item Description of the event handlers and subroutines.
\item Programming notes.
\item Experiments for you to do.
\end{itemize}

\centeredbox{I strongly suggest that you work on each project by
yourself and only then compare your solution to mine.}

\newpage

\sect{State machines}

You stand in front of a vending machine and wait for a candy bar to be
pushed out. Nothing happens. You push a button to select which candy you
want. Nothing happens. Finally, you put some coins in the machine.
Nothing happens. Again, you push a button to make a selection. Now, the
machine starts working and the candy is pushed out.

A vending machine is really a robot; it's task is to give candy to
people. The task of a vending machine is divided into
smaller subtasks and only one subtask is active at any time. We say that
the machine can be in several \emph{states} and that there are
conditions that determine when the machine makes a \emph{transition}
from one state to the next. The states and transitions can be described
in the diagram of a \emph{state machine}, where the states are the ovals
and the transitions are the arrows:

\begin{center}
\unitlength=1.4pt
\begin{picture}(280,45)
%\put(0,0){\framebox(280,45){}}
\put(30,10){\oval(60,20)}
\put(140,10){\oval(60,20)}
\put(250,10){\oval(60,20)}
\put(0,0){ \makebox(60,20){\shortstack{\bu{expect}\\\bu{coins}}}}
\put(110,0){\makebox(60,20){\shortstack{\bu{expect}\\\bu{selection}}}}
\put(220,0){\makebox(60,20){\shortstack{\bu{wait for}\\\bu{candy}}}}
\put( 60,10){\vector(1,0){50}}
\put( 58,10){\makebox(48,10){\bu{put coin}}}
\put(170,10){\vector(1,0){50}}
\put(171,10){\makebox(48,10){\bu{press button}}}
\put(250,20){\line(0,1){15}}
\put(30,35){\line(1,0){220}}
\put(30,35){\makebox(220,10){\bu{candy pushed}}}
\put(30,35){\vector(0,-1){15}}
\end{picture}
\end{center}

The state machine makes it clear why you won't receive any candy just by
standing in front of the vending machine. Candy is pushed out only in
the rightmost state which can be reached only after putting coins and
making a selection.

All our projects use state machines to describe the behavior of the
robots. The state machine can be as simple as:

\begin{center}
\unitlength=1.4pt
\begin{picture}(180,45)
%\put(0,0){\framebox(180,45){}}
\put(30,10){\oval(60,20)}
\put(150,10){\oval(60,20)}
\put(0,0){ \makebox(60,20){\bu{off}}}
\put(120,0){\makebox(60,20){\bu{on}}}
\put( 60,10){\vector(1,0){60}}
\put( 60,10){\makebox(60,20){\shortstack{\bu{press}\\\bu{center button}}}}
\put(150,20){\line(0,1){15}}
\put(30,35){\line(1,0){120}}
\put(30,35){\makebox(120,10){\bu{press center button}}}
\put(30,35){\vector(0,-1){15}}
\end{picture}
\end{center}

but usually it will be more complex because the robot---like a vending
machine---must perform several subtasks.

\newpage

\sect{Overview of the projects}

\begin{description}

\item[Chapter \ref{ch.cat}: The cat catches the mouse] The robot is a
cat which searches for and catches a mouse. This type of requirement is
common when a robot has to locate an object and move it someplace.

\item[Chapter \ref{ch.fast}: How fast is the Thymio?] The
performance of the robot is measured. We compute the speed and
acceleration of the robot, and the relationship between the amount of
power applied to the motors and the acceleration of the robot.

\item[Chapter \ref{ch.line}: Line following] An autonomous robot must be
able to \emph{navigate}, that is, to move from its current position to a
new position, for example, by following a line displayed on the floor.

\item[Chapter \ref{ch.control}: Controlling the robot] An autonomous
robot must make decisions, such as in what direction to turn and how
fast to travel. The robot uses control algorithms such as proportional
control to make these decisions.

%\item[Chapter \ref{ch.fuzzy}: Fuzzy logic] The algorithms in the
%previous chapter used precise mathematical computations to control the
%robot. Fuzzy logic uses imprecise language to specify control
%algorithms.

\item[Chapter \ref{ch.nonlinear}: Nonlinear sensors] A sensor is linear
if the values it returns are a linear function of what it measures. For
example, a linear distance sensor might return the value $50\cdot d$,
where $d$ is the distance to the object. It will return 50 if an object
is 1 meter away and 100 if the object were 2 meters away. The
nonlinearity of a sensor can be measured and used to calibrate the
sensor.

\item[Chapter \ref{ch.music}: Composing music on-the-fly] The Thymio
contains accelerometers which can be used to measure both the
attitude of the robot (because gravity is an acceleration that always
points downwards) and to sense if the robot is hit or shaken. This
project uses the accelerations obtained when the robot is shaken to
compose music.

\item[Chapter \ref{ch.odo}: Introduction to odometry and Chapter
\ref{ch.odo-advanced}: Advanced odometry] Odometry is the calculation of
a robot's position and heading obtained by starting at an initial
position and using the measured velocity and time. These projects show
how this is done, first for movement in a straight line and then in a
turning movement.

\item[Chapter \ref{ch.barcode}: Image recognition] A robot must sense
its environment in order to carry out its tasks. This project
demonstrates how a robot can search for a particular object by
recognizing its barcode.

\end{description}

\newpage

\sect{Increasing the detection range of the sensors}

\centeredbox{Ordinary objects need to be very close to the Thymio before they are
detected by the horizontal proximity sensors. You can greatly increase
the range by attaching \emph{reflecting tape}, such as used on bicycles,
to the objects.\footnote{My thanks to Francesco Mondada for showing this
to me!} The settings of the thresholds and motor powers in the programs
in the archive will, of couse, have to be changed.\label{p.reflect}}

\sect{References}

\begin{itemize}
\item Visual programming and VPL:
\href{https://www.thymio.org/en:visualprogramming}{https://www.thymio.org/en:visualprogramming}.

\item Text programming: the AESL language, the Studio environment and the Thymio interface:
\href{https://www.thymio.org/en:asebausermanual}{https://www.thymio.org/en:asebausermanual}.
\end{itemize}
