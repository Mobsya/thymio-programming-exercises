% !TeX root = aseba.tex

\chap{Image Recognition}\label{ch.barcode}

A robot must be able to look at its environment in order to locate
and identify the targets of its actions or objects that may be in its way.
Image recognition normally requires a camera;
this project will use the proximity sensors.

Consider a robot in a warehouse where all objects are identified by barcodes.
Let us see how a robot could search for a specific object
in an aisle with many objects.

\sect{Specification}

The Thymio moves straight and at frequent intervals turns left to identify
to object on a shelf. After sampling the objects,
it turns back right and then continues straight.
When the robot locates the object, it emits a signal.

\sect{Design}

{\raggedleft \hfill Program file \bu{barcode.aesl}}

Take a piece of wood about the height of the robot and about 50 cm long.
Cover it with black tape so the wooden background will not be detected
by the sensors. The objects are represented by strips of aluminum foil
fastened over the black tape. Narrow strips are objects we don't want
and the wide strip is the object we want.

\gr{barcode}{.9}

The robot computes the sum of the values returned by the
five horizontal sensors.
When this value is above a threshold, the robot detects
the wide strip and the top LEDs are turned on.
All the sums of the samples are stored in an array
so that they can be examined after the robot stops.

\sect{State machine}

The program transitions sequentially through four states:

\begin{center}
\unitlength=1.2pt
\begin{picture}(300,35)
%\put(0,0){\framebox(300,35){}}
\put(30,10){\oval(60,20)}
\put(110,10){\oval(60,20)}
\put(190,10){\oval(60,20)}
\put(270,10){\oval(60,20)}
\put(0,0){ \makebox(60,20){\bu{0=off}}}
\put(80,0){\makebox(60,20){\bu{1=straight}}}
\put(160,0){\makebox(60,20){\bu{2=left}}}
\put(240,0){\makebox(60,20){\bu{3=right}}}
\put( 60,10){\vector(1,0){20}}
\put(140,10){\vector(1,0){20}}
\put(220,10){\vector(1,0){20}}
\put(270,20){\line(0,1){15}}
\put(110,35){\line(1,0){160}}
\put(110,35){\vector(0,-1){15}}
\end{picture}
\end{center}

\begin{itemize}

\item The transition from \bu{off} to \bu{straight} occurs when
the forward button is touched.

\item The transition from \bu{straight} to \bu{left} occurs
when \p{TIME\_STRAIGHT} has passed.

\item The transition from \bu{left} to \bu{right} occurs
when \p{TIME\_LEFT} has passed.

\item The transition from \bu{right} to \bu{straight} occurs
when \p{TIME\_RIGHT} has passed.

\item In addition to the transitions shown in the diagram,
touching the center button causes a transition to state \bu{off},
as does detecting the end of the table.
\end{itemize}

\sect{Constants}

\begin{itemize}

\item \p{MOTOR}: The power of the motors.

\item \p{THRESHOLD}: The threshold for detection of the wide aluminum strip.

\item \p{TIME\_STRAIGHT}, \p{TIME\_LEFT}, \p{TIME\_RIGHT}: Values for
the timer to control the search and the turns.
Turning left and right should take the same amount of time,
but inconsistencies may require the use of slightly different times.

\end{itemize}

\sect{Variables}

\begin{itemize}
\item \p{state}: The current state. 
\item \p{samples}: An array to store the sums of the samples.
\item \p{i}: Index for sampling all the sensors.
\item \p{j}: Index for the array \p{samples}.
\end{itemize}

\sect{Event handlers and subroutines}

The events handlers for touching the center and forward buttons and for detecting
the edge of the table are straightforward. Most of the computation is done
in the handler for \p{timer0}. It contains an \p{if}-statement
with four alternatives, one for each state. Each alternative changes the
motor power as required and sets the timer for the next state.

Before turning right in state 2, the sensors are sampled.
If the sum is greater than \p{THRESHOLD}, the top LEDs are turned on.

\sect{Running the program}

It is quite difficult to get this program to run correctly,
because turning left and then right may not return the robot
to exactly the same heading. If the table surface if not perfectly
smooth, these deviation can change as the robot moves along the piece of wood.

\sect{Experiments}

\begin{enumerate}

\item Experiment with the speed of the robot. How fast can you get it to
search for the object and still identify it.

\item See if the program can identify multiple objects (wide strips).

\item Measure how far the robot moves for each sample.
Modify the program to count the number of samples until it finds the object
and then to compute to distance from the starting point to the object.

\item Experiment with the time between samples (\p{TIME\_STRAIGHT}).
More frequent samples will cause the robot to take a longer time
to locate the object; however, it will locate its position
with greater accuracy.

\item The accuracy of the search can be improved by
placing a tape on the surface and using a line-following program
(Chapter~\ref{ch.line}). This will not only ensure that the robot travels
straight, but it will make it easy to determine when the right turn
has moved to robot back to its initial heading.

\end{enumerate}

\newpage

\sect{Project}

Develop a program that reads a barcode consisting of three strips of
tape placed on a table. The strips are detected by the ground sensors
and interpreted as encodings of RGB colors. When the strips have been
decoded, the top LEDS will display that color.

{\raggedleft \hfill Program file \bu{color-code.aesl}}

\textbf{Guidance:}

Place three strips of tape of varying widths on a table.

\begin{center}
\begin{picture}(240,50)
%\put(0,0){\framebox(240,50){}}

\put(0,0){\framebox(30,30){}}
\put(30,15){\oval(30,30)[r]}
\put(45,15){\vector(1,0){15}}

\put(80,0){
\put(0,  0){\colorbox{black}{\makebox(10,30){}}}
\put(50, 0){\colorbox{black}{\makebox(20,30){}}}
\put(110,0){\colorbox{black}{\makebox(40,30){}}}

\put(0,  40){\makebox(10,10)[c]{$w_1$}}
\put(50, 40){\makebox(20,10)[c]{$w_2$}}
\put(110,40){\makebox(40,10)[c]{$w_3$}}
}

\end{picture}
\end{center}

Define two constants \p{MOTOR\_SPEED} and \p{THRESHOLD}. The first is
the power setting for both the left and right motors, and the second is
the value of the ground sensors below which the black tape is detected.
(Should you use the value of the left ground sensor, the right one or
some combination of the two?)

The Thymio travels across the strips of tape. When it detects a black
tape during a \p{prox} event, it zeros a counter. The counter is
incremented during each subsequent \p{prox} event until the black tape
is no longer detected. The counter is then stored in an array. When
three strips have been detected, the three values stored in the array
are used to compute the values of the red, green and blue components and
\p{leds.top} is set accordingly. The higher the value of the counter
(that is, the more events that occurred while moving over a strip), the
wider the strip and the higher the value of the color component.

You will have to experiment with the parameters of this project to get
it to work properly: the width of the strips of tape, the constants
\p{MOTOR\_SPEED} and \p{THRESHOLD}, and the mapping between counter
values and the values of the color components.

Run the program with different arrangements of the strips of tape in order to
display different colors.
