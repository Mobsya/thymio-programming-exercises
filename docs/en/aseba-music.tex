% !TeX root = aseba.tex

\chapter{Composing Music On-the-Fly}

The Thymio contains \emph{accelerometers} that measure
the acceleration of the robot on each of the three axes: left-right,
front-back and top-down. When the robot is at rest, the acceleration of
the left-right and front-back axes will be zero, but the accelerometer
for the top-down axis will measure a positive acceleration corresponding
to the earth's gravity.
The accelerometers are used to detect shocks (like the \emph{tap} event)
and they can be used to detect the orientation of the robot relative
to the ground. The project in this chapter uses movement of the robot
to compose music.

\sect{Specification}

Hold the Thymio in your hand and move it quickly left and right.
The program will sample the left-right acceleration and store the samples.
When enough samples have been taken, their values will be used to play
music using the sound synthesizer.

\sect{System design}

Before continuing, review sections \emph{Accelerometer}
and \emph{Synthetic sound} of the \emph{Programming Interface}
at \url{https://aseba.wikidot.com/en:thymioapi}.


\sect{State machine}

There are two states: 0 for off and 1 for on.

\sect{Constants}

\begin{itemize}
\item \p{SIZE}: The number of samples of the acceleration.
\item \p{PERIOD}: The number of milliseconds between samples.
\item \p{BASE}: The base frequency for sound.
\item \p{FACTOR}: The factor for changing the frequency of each sound.
\item \p{DURATION}: The duration of each sound.
\end{itemize}

\sect{Variables}

\begin{itemize}
\item \p{state}: The state: on or off. 
\item \p{save}: An array of length \p{SIZE} for saving the samples.
\item \p{i}: An array index variable.
\end{itemize}

\sect{Event handlers and subroutines}

\begin{itemize}
\item Event \p{button.center}: When the center button is released,
set \p{state} to 1 and initialize the array index variable \p{i} to 0.
\item Event \p{timer1}: When the timer expires (and \p{state} is 1),
save the value of \p{acc[0]} in the array \p{save} and increment
the index. When the array is full, set \p{state} to 0 and initiate
a sound by calling \p{sound.freq}.
\item Event \p{sound.finished}: When a sound is finished,
initiate the next sound by calling \p{sound.freq}.
\end{itemize}

\sect{Programming notes}

{\raggedleft \hfill Program file \bu{music1.aesl}}

\begin{itemize}
\item To initialize the array elements to zero, use the native function \p{math.fill}.
\item Initialize the timer to \p{PERIOD}.
\item The values of the acceleration are between $-32$ and $32$,
so multiply them by \p{FACTOR} and add to \p{BASE} to obtain
an audible note (several hundred hertz):
\begin{verbatim}
    call sound.freq(BASE+save[i]*FACTOR, DURATION)
\end{verbatim}
The value of the duration is the number of $\frac{1}{60}$ths of a second.
\end{itemize}

\sect{Experiments}

\begin{itemize}
\item Experiment with the values of the constants. Make sure that
the note played---the first parameter
to \p{sound.freq}---doesn't go below or above a sound you can hear.

\item The sounds obtained are not ``real'' notes but just sounds of
different frequencies. Modify the program so that only real notes are played.
Start by defining an array of frequencies; for example, the frequencies
from \emph{middle C} to \emph{high C}, rounded to integers, are:
\begin{verbatim}
var notes[8] = [261, 294, 330, 349, 392, 440, 494, 523]
\end{verbatim}
The call to initiate a sound is now:
\begin{verbatim}
call sound.freq(notes[...], DURATION)
\end{verbatim}
Compute an index to \p{notes} in the range 0--7, based upon the
values in \p{save}.

Hint: Compute a value and take its remainder when divided by 8 using the modulo operator \verb+%+.

{\raggedleft \hfill Program file \bu{music2.aesl}}

\item Modify the program to store the left-right acceleration and
the forward-back acceleration in two arrays. Use the left-right value
to compute the note and the forward-back value to compute the duration.

{\raggedleft \hfill Program file \bu{music3.aesl}}

\item Try replacing the forward-back acceleration with the top-bottom
acceleration. Explain what happens.
\end{itemize}