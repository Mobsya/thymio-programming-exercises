% !TeX root = aseba.tex

\chapter{Composing Music On-the-Fly}\label{ch.music}

The Thymio contains \emph{accelerometers} that measure
the acceleration of the robot on each of the three axes: left-right,
front-back and top-down. When the robot is at rest, the acceleration of
the left-right and front-back axes will be zero, but the accelerometer
for the top-down axis will measure a positive acceleration corresponding
to the earth's gravity.
The accelerometers are used to detect shocks (like the \emph{tap} event)
and they can be used to detect the orientation of the robot relative
to the ground. The project in this chapter uses movement of the robot
to compose music.

\sect{Specification}

Hold the Thymio in your hand and move it quickly left and right.
The program will sample the left-right acceleration and store the samples.
When enough samples have been taken, their values will be used to play
music using the sound synthesizer.

\sect{State machine}

\begin{center}
\unitlength=1.2pt
\begin{picture}(300,35)
%\put(0,0){\framebox(300,35){}}
\put(30,10){\oval(60,20)}
\put(110,10){\oval(60,20)}
\put(190,10){\oval(60,20)}
\put(270,10){\oval(60,20)}
\put(0,0){ \makebox(60,20){\bu{0=off}}}
\put(80,0){\makebox(60,20){\bu{1=sample}}}
\put(160,0){\makebox(60,20){\bu{2=stop}}}
\put(240,0){\makebox(60,20){\bu{3=play}}}
\put( 60,10){\vector(1,0){20}}
\put(140,10){\vector(1,0){20}}
\put(220,10){\vector(1,0){20}}
\put(270,20){\line(0,1){15}}
\put(30,35){\line(1,0){240}}
\put(30,35){\vector(0,-1){15}}
\end{picture}
\end{center}

The initial state is \bu{off}. When the center button is touched,
the transition to \bu{sample} is taken and the robot begins to sample
the accelerations. The samples are stored in a data structure; when
it is full, the transition to \bu{stop} is taken. Touching the center button again
causes a transition to \bu{play} and the sounds to be played. When all the
sounds have been played (or the center button is touched again), the
transition to state \bu{off} is taken.

\sect{Constants}

\begin{itemize}
\item \p{SIZE}: The number of samples of the acceleration.
\item \p{PERIOD}: The number of milliseconds between samples.
\item \p{BASE}: The base frequency for sound.
\item \p{FACTOR}: The factor for changing the frequency of each sound.
\item \p{DURATION}: The duration of each sound.
\end{itemize}

\sect{Variables}

\begin{itemize}
\item \p{save}: An array of length \p{SIZE} for saving the samples.
\item \p{index}: An array index variable.
\item \p{state}: The state.
\end{itemize}

\sect{Event handlers and subroutines\footnote{Before continuing,
review sections \emph{Accelerometer} and \emph{Synthetic sound} of the \emph{Programming Interface} at \url{https://aseba.wikidot.com/en:thymioapi}.}}

\begin{itemize}

\item Event \p{button.center}: When the center button is released,
the state transitions described above are performed.
During the transition from state \bu{stop} to state \bu{play} a short
sound is played so that the event \bu{sound.finished} will occur.
During the transition from state \bu{play} to state \bu{off},
\bu{sound.freq} is called with a duration of $-1$ to stop the sound.
In all cases, the variable \p{index} is set to 0.

\item Event \p{timer1}: When the timer expires in state \bu{sample},
the value of \p{acc[0]} is stored in \p{save} and \p{index} is incremented.
When the array is full, the state is set to \bu{stop}.

\item Event \p{sound.finished}: When a sound is finished,
initiate the next sound by calling \p{sound.freq}. When all sounds
have been played, go to state \bu{off}.
\end{itemize}

\sect{Programming notes}

{\raggedleft \hfill Program file \bu{music1.aesl}}

\begin{itemize}
\item To initialize the array elements to zero, use the native function \p{math.fill}.
\item Initialize the timer to \p{PERIOD}.
\item The values of the acceleration are between $-32$ and $32$,
so multiply them by \p{FACTOR} and add to \p{BASE} to obtain
an audible note (several hundred hertz):
\begin{verbatim}
    call sound.freq(BASE+save[index]*FACTOR, DURATION)
\end{verbatim}
The units of DURATION are $\frac{1}{60}$ths of a second.
\end{itemize}

\sect{Experiments}

\begin{itemize}
\item Experiment with the values of the constants. Make sure that
the note played---the first parameter
to \p{sound.freq}---doesn't go below or above a sound you can hear.

\item Set the LEDs on the top of the robot to reflect its current state.

{\raggedleft \hfill Program file \bu{music2.aesl}}


\item The sounds obtained are not ``real'' notes but just sounds of
different frequencies. Modify the program so that only real notes are played.
Define an array of frequencies: the frequencies
from \emph{middle C} to \emph{high C}, rounded to integers, are:
\begin{verbatim}
var notes[8] = [261, 294, 330, 349, 392, 440, 494, 523]
\end{verbatim}
Use the values in \p{save} to compute an index to \p{notes} to select
a frequency for \p{call sound.freq(notes[...], DURATION)}.
Use the modulo operator \p{\%\ 8} to ensure that the index is in
the range 0--7.

{\raggedleft \hfill Program file \bu{music3.aesl}}

\item In addition to the left-right acceleration,
save the forward-back acceleration in another array. Use the left-right values
to compute the note and the forward-back values to compute the duration.
Explain what happens if you use top-bottom acceleration instead.

{\raggedleft \hfill Program file \bu{music4.aesl}}

\item Instead of calling \p{leds.top} whenever the state is changed, write an event handler that checks the current value of \p{state} and sets the LEDs accordingly. Since the \p{prox} event is not otherwise used in this project
and occurs frequently (10 times a second), it is convenient to use it for this purpose.

{\raggedleft \hfill Program file \bu{music5.aesl}}

\item By default, one circle LED is lit at the lowest point of the robot.
Modify this so that two circle LEDs are lit: left or right and front or back to show the directions of the accelerations in these axes.

{\raggedleft \hfill Program file \bu{music6.aesl}}

\end{itemize}